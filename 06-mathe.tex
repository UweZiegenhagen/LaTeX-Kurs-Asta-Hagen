\documentclass[15pt,ngerman]{scrreprt}

\usepackage[utf8]{inputenc}
\usepackage[T1]{fontenc}
\usepackage{babel}

\usepackage[]{esvect}
\usepackage[]{nicefrac}
% für michas fotoalbum

%\usepackage[landscape,left=1cm,right=1cm,bottom=2.5cm, top=1cm]{geometry}

\usepackage{subcaption}

% Beispieldefinition für \person{}

\newcommand{\person}[1]{\textsc{#1}}

\author{Uwe Ziegenhagen}
\title{Mein allererstes \LaTeX-Dokument}
%\date{1. Januar 2018}

\usepackage{blindtext}

\setlength{\parindent}{0pt}
\setlength{\parskip}{10pt}

\usepackage{microtype}
\usepackage{graphicx} % Bilder

\usepackage{paralist} %kompakte Aufzählungen

\usepackage{hyperref} % immer als letztes Paket

\hypersetup{
    bookmarks=true,                     % show bookmarks bar
    unicode=false,                      % non - Latin characters in Acrobat’s bookmarks
    pdftoolbar=true,                        % show Acrobat’s toolbar
    pdfmenubar=true,                        % show Acrobat’s menu
    pdffitwindow=false,                 % window fit to page when opened
    pdfstartview={FitH},                    % fits the width of the page to the window
    pdftitle={My title},                        % title
    pdfauthor={Author},                 % author
    pdfsubject={Subject},                   % subject of the document
    pdfcreator={Creator},                   % creator of the document
    pdfproducer={Producer},             % producer of the document
    pdfkeywords={keyword1, key2, key3},   % list of keywords
    pdfnewwindow=true,                  % links in new window
    colorlinks=true,                        % false: boxed links; true: colored links
    linkcolor=blue,                          % color of internal links
    filecolor=blue,                     % color of file links
    citecolor=blue,                     % color of file links
    urlcolor=blue                        % color of external links
}

\usepackage[]{stix}

\makeatletter
\newcommand*\avg{%
\mathop{\operator@font
avg}}
\makeatother


\begin{document}
\maketitle

\tableofcontents

\listoffigures

\chapter{Das allererste Kapitel, los geht's}

\section{Einleitung des ersten Dokuments, was ich wo geschrieben hab}

\blindtext

Siehe Abbildung \ref{fig:ente} auf Seite \pageref{fig:ente}.

\section{Hauptteil}

\subsection{Einleitung zum Hauptteil}

Irgendein Text.

\subsubsection{Hauptteil}


\blindtext[2]

\blindtext[2]

\section{Mathesatz}

$a^2$ ist \TeX-Notation.

Abc \( a^2 + b^2 = c^2\) def

Abc \[ a^2 + b^2 = c^2\] def % $$ a^2 + b^2 = c^2$$


\begin{equation}
a^2 + b^2 = c^2
\end{equation}

\begin{equation}
\frac{a^2 + b^2 = c^2}{a^2 + b^2 = c^2}
\end{equation}

\begin{equation}
-\frac{p}{2} \pm \sqrt{ \left(\frac{p}{2}\right)^2 - q  }
\end{equation}


\begin{equation}
\sum_{i=1}^{\infty} i
\end{equation}

\begin{equation}
\prod_{i=1}^{\infty} i
\end{equation}

\begin{equation}
\bigtimes_{i=1}^{\infty} i
\end{equation}


\begin{equation}
\overbrace{\bigtimes_{i=1}^{\infty} i}^{a\times b}
\end{equation}

\begin{equation}
sin 45 \not= \sin 45 \avg
\end{equation}

\begin{eqnarray}
y &=& (a+ b)^2 \\
  &=& a^2 + 2ab + b^2
\end{eqnarray}

\begin{eqnarray*}
y &=& (a+ b)^2 \\
  &=& a^2 + 2ab + b^2
\end{eqnarray*}

\section*{Ich bin nicht im ToC}

\[
\bordermatrix{%
 & 0 & 1 & 2 \cr
0 & A & B & C \cr
1 & d & e & f \cr
2 & 1 & 2 & 3 \cr
}
\]

\( a_{111}+ b_1 =  \vec{c_1} \times \vec{c}_1  \times \vv{c_1}   \)

\( \alpha + \beta = \gamma\)

\( \alpha + \beta = \Gamma\)


Fließtext-Integrale sind kleiner \( \int_{i=1}^{\infty} \)

\[ \int_{i=1}^{\infty} \]

\[ \nicefrac{a}{2}\]

\end{document}
