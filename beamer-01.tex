\documentclass[12pt,ngerman]{beamer}

\usepackage[utf8]{inputenc}
\usepackage[T1]{fontenc}
\usepackage{babel}
\usetheme{PaloAlto}

\title{Meine erste Präsentation}
\author{Uwe Ziegenhagen}
\institute{FernUni Hagen}
%\logo{\includegraphics[width=2cm]{./Bilder/image1}}

\begin{document}

\begin{frame}

\maketitle

\end{frame}

\begin{frame}

\tableofcontents

\end{frame}


\section{Einleitung}

\begin{frame}
\frametitle{Meine erste Folie}

\begin{itemize}
	\item Hallo
	\item Welt
	\item Foo
 	\item Bar
	\item Schnick
	\item Schnack
\end{itemize}

\end{frame}

\section{Hauptteil}

\begin{frame}
\frametitle{Meine zweite Folie}

\begin{enumerate}
	\item Hallo
	\item Welt
	\item Foo
 	\item Bar
	\item Schnick
	\item Schnack
\end{enumerate}

\end{frame}

\begin{frame}
\frametitle{Mathe}

\begin{equation}
x_{1,2} = -\frac{p}{2} \pm \sqrt{\left(\frac{p}{2} \right)^2-q}
\end{equation}


\end{frame}


\begin{frame}
\frametitle{There Is No Largest Prime Number}
\framesubtitle{The proof uses \textit{reductio ad absurdum}.}
\begin{theorem}
There is no largest prime number.
\end{theorem}
\begin{proof}
\begin{enumerate}
\item<1-| alert@1> Suppose $p$ were the largest prime number.
\item<2-> Let $q$ be the product of the first $p$ numbers.
\item<3-> Then $q+1$ is not divisible by any of them.
\item<1-> But $q + 1$ is greater than $1$, thus divisible by some prime
number not in the first $p$ numbers.\qedhere
\end{enumerate}
\end{proof}
\end{frame}

\end{document}